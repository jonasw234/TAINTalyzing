\chapter{Evaluierung}
    Im Folgenden werden die in dieser Masterarbeit erreichten Ergebnisse zusammengefasst,
    Empfehlungen für den Einsatz des entwickelten Frameworks vorgestellt und
    es wird ein Ausblick gegeben auf mögliche weitere Entwicklungen des Frameworks,
    die aufgrund der zeitlichen Begrenzung und
    dem Universalitätsgedanken nicht behandelt werden konnten.

    \section{Erreichte Ergebnisse}
        In dieser Ausarbeitung wurde ein Framework erstellt,
        welches zur statischen Analyse von Quelltexten dient,
        um Sicherheitslücken zu entdecken.

        Das Besondere an diesem Framework ist,
        dass versucht wurde,
        es möglichst universell einsetzbar zu gestalten,
        sodass es späteren Anwendern möglichst leicht sein sollte,
        das Framework nicht nur mit eigenen Regeln,
        sondern auch mit eigenen Grammatiken für gänzlich neue Programmiersprachen zu erweitern.

        Als Beispiele wurden drei Module für Programmiersprachen implementiert,
        die jeweils aus den Grammatiken zur Beschreibung der Syntax sowie
        aus den Regeln zur Beschreibung der Quellen,
        Senken und
        Absicherungen bestehen.

        Die Grammatik für C ist hierbei die ausgereifteste,
        mit der die meisten Syntaxelemente gut erkannt werden können.

        Bei der Grammatik für
        \gls{PHP} soll vor allem gezeigt werden,
        wie Grammatiken für objektorientierte Programmiersprachen erstellt werden können.

        Die Grammatik für Python hat als Besonderheit,
        dass in Python nicht geschweifte Klammern,
        sondern Einrückungen zur Markierung von Codeblöcken benutzt werden.

        Mit allen drei Grammatiken ist es möglich,
        nicht nur Schwachstellen in Quelltexten aufzudecken,
        sondern diese Schwachstellen auch über mehrere Funktionsaufrufe hinweg zu verfolgen und
        mit anderen Funktionen zu verknüpfen,
        wie jeweils mit Testfällen in der Datei
        \filename{test.py} nachgewiesen wurde.

        Dies zeigt einen weiteren Vorteil des Frameworks: Es ist nicht notwendig,
        vollständige Grammatiken für eine Sprache zu erstellen,
        die jedes einzelne Konstrukt genau verstehen.
        Stattdessen lassen sich unvollständige und
        damit auch robustere Grammatiken erstellen,
        mittels derer bereits einige Sicherheitslücken gefunden werden können.

        Ob die Benutzer letztlich mehr Wert auf Genauigkeit bei der Erstellung der Grammatiken legen,
        was mit höherem Arbeitsaufwand und
        geringerer Robustheit gegenüber Änderungen an der Syntax der Programmiersprache einherginge,
        bleibt ihnen überlassen.

        Es ist für jedes Modul möglich,
        mittels einfacher Regeln Funktionen zu beschreiben,
        welche als Benutzereingabe,
        potenziell verwundbare Funktion oder
        als Absicherung einer verwundbaren Funktion dienen.

        Im Anschluss untersucht das Framework dann selbstständig die angegebenen Pfade auf diese verwundbaren Funktionen und
        gibt an,
        ob diese mit potenziell benutzerkontrollierten Eingaben aufgerufen werden,
        was zu einer Kompromittierung des Systems führen könnte.

        Durch die Ausgabe des Reports in verschiedenen Formaten ist eine einfache Weiterverarbeitung und
        Anpassung möglich,
        was besonders den Einsatz im Geschäftsumfeld erleichtern sollte.

    \section{Empfehlungen}
        Das erstellte Framework kann zwar dabei helfen,
        Sicherheitslücken in Quelltexten zu finden,
        bietet aber keine absolute Sicherheit,
        dass alle Schwachstellen entdeckt werden.

        Dies liegt zum einen daran,
        dass die Schwachstellensuche stark von der Anzahl und
        Qualität der genutzten Regelsätze abhängt,
        zum anderen aber auch,
        weil es mittels statischer Analyse nicht möglich ist,
        in jedem Fall alle Schwachstellen in einem Quelltext aufzudecken,
        wie bereits in
        \vref{Satz von Rice} erläutert wurde.

        Stattdessen sollten Anwender zusätzlich immer auch eine manuelle Analyse durchführen und
        die Resultate des Frameworks manuell prüfen.

        Es kann jedoch sehr hilfreich sein,
        automatisierte Sicherheitsscans,
        wie sie dieses Framework bietet,
        in den Entwicklungsprozess mit einzubeziehen,
        um Schwachstellen möglichst frühzeitig und
        kostensparend zu entdecken und
        zu beheben.

        Da das Framework auch in einer Docker"=Umgebung läuft,
        sollte es immer möglich sein,
        das Framework auf Entwicklungsservern einzusetzen.

    \section{Ausblick}
        Da das Framework theoretisch beliebige Programmiersprachen unterstützt,
        sollten im nächsten Schritt die Programmiersprachen ermittelt werden,
        die in der Praxis zum einen am häufigsten eingesetzt werden und
        zum anderen am meisten von einem automatisierten Sicherheitsscan profitieren.

        Anschließend sollten Grammatiken und
        Regelsätze für diese Programmiersprachen erstellt werden.

        Nach Möglichkeit sollten diese Grammatiken und
        Regeln in einem zentralen Repository gespeichert werden,
        sodass später nicht mehr jeder Anwender seine eigenen Grammatiken und
        Regeln schreiben muss,
        sondern auf der Arbeit von anderen aufbauen kann,
        wodurch sich die Qualität der Analyse stetig verbessert.

        Die statische Analyse alleine kann nicht sämtliche Sicherheitslücken in einem System finden,
        allerdings ist es hiermit möglich,
        auch selten manuell geprüfte Codezweige abzudecken,
        sodass sie nach Möglichkeit immer zusätzlich zur dynamischen Analyse eingesetzt werden sollte.

        Mit dem hier vorgestellten Framework sind Benutzer somit in der Lage,
        bereits mit wenig Aufwand ihre Programme sicherer zu gestalten.

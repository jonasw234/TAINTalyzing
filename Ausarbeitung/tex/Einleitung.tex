\chapter{Einleitung}
    \section{Motivation}
        Das Aufspüren und
        Beheben von Sicherheitslücken ist eine wichtige Aufgabe bei der Entwicklung von Software.

        Zwar existieren auf den meisten heutigen Systemen Schutzmechanismen,
        die ein Ausnutzen vorhandener Sicherheitslücken erschweren sollen,
        doch können diese natürlich nicht als Allheilmittel eingesetzt werden,
        da sie zum einen nicht auf allen Systemen vorhanden und
        aktiv sind,
        zum anderen aber auch,
        da sie nicht das darunter liegende Problem selbst korrigieren,
        sondern nur das Ausnutzen erschweren sollen und
        sich somit meistens umgehen lassen.

        Dies kann dazu führen,
        dass häufig nicht mehr einzelne Lücken von Angreifern ausgenutzt werden,
        um Systeme zu übernehmen,
        sondern es werden mehrere Lücken miteinander kombiniert,
        um Stück für Stück die Abwehrmaßnahmen zu umgehen und
        Angriffsmöglichkeiten auszuweiten.

        Eine manuelle Analyse von Quelltexten auf Sicherheitslücken dagegen ist sowohl
        mit hohen Kosten verbunden,
        da extra geschulte Mitarbeiter zu diesem Zweck in mühevoller Arbeit Quelltexte analysieren müssen,
        als auch selbst wiederum fehleranfällig,
        da gerade bei komplexen Programmabschnitten der Programmfluss unter Umständen sehr kompliziert wird.

        Aus diesem Grund kann eine statische Analyse von Quelltexten im Hinblick auf Schwachstellen hilfreich sein,
        um verschiedene Fehler schnell und
        einfach zu finden.

        Natürlich kann eine statische Analyse nicht sämtliche Fehler finden,
        allerdings kann die Zeit,
        die Mitarbeiter mit der Analyse von Quelltexten verbringen,
        sinnvoller dafür eingesetzt werden,
        komplexe und
        fehleranfällige Codestellen intensiver zu prüfen,
        als den kompletten Quelltext minutiös nach trivialen Schwachstellen abzusuchen.

        Aus diesem Grund sollte eine manuelle Analyse gerade auch komplexer Codeabschnitte erfolgen,
        wobei auf diese besonders hingewiesen werden sollte.

        Da immer wieder neue Muster entdeckt werden,
        welche einen Angriff ermöglichen,
        ist es dabei besonders wichtig,
        dass die Regeln für die statische Analyse möglichst einfach erweiterbar sind.

        Viele existierende Schwachstellenscanner haben dieses Problem,
        da sie zum einen auf eine kleine Anzahl von Programmiersprachen beschränkt sind,
        zum anderen aber auch häufig entweder einen fest einprogrammierten Regelsatz haben oder
        keine einfache Anpassung und Erweiterung der Regelsätze durch Benutzer erlauben.

    \section{Problemstellung und -abgrenzung}\label{Problemstellung und -abgrenzung}
        Die Aufgabe besteht darin,
        ein Framework zu erstellen,
        welches der statischen Analyse von Quelltexten dient.

        Das Framework soll dabei möglichst flexibel sein,
        um eine große Anzahl von Quelltexten untersuchen zu können und
        eine stetige Anpassung und
        Erweiterung im Unternehmensumfeld zu ermöglichen.

        Zu diesem Zweck wird das Framework modular aufgebaut,
        wobei durch jedes Modul eine andere Programmiersprache analysiert werden kann.
        Somit soll es möglich sein,
        mit nur einem einheitlichen Framework sämtliche Quelltexte automatisch auf Schwachstellen untersuchen zu können.

        Dabei könnte ein Modul für die Erkennung und
        Analyse von Quelltexten in C,
        ein anderes für die Erkennung und
        Analyse von Quelltexten in
        \gls{PHP} oder
        Python verantwortlich sein.
        Die Modularisierung hilft dabei,
        die Komplexität gering zu halten und
        gewährleistet eine einfachere Erweiterbarkeit,
        da Benutzer auch selbst neue Module schreiben und
        zum Scannen verwenden können.

        Innerhalb der Module wiederum werden Regelsätze verwendet,
        anhand derer Schwachstellen beschrieben werden.

        Hierbei werden in den Regelsätzen verwundbare Funktionen und
        auch Benutzereingaben aufgelistet,
        sodass Anwender später nachvollziehen können,
        an welcher Stelle ihr Quelltext verwundbar sein könnte und
        ob ein direkter Angriff durch eine Benutzereingabe erkannt wurde.

        Anschließend wird das Framework versuchen automatisch zu erkennen,
        welches Modul es für welche Datei aufrufen soll,
        und
        anhand der Regelsätze die Quelltexte untersuchen.

        Es wird dabei nicht darauf eingegangen,
        ob ein Ausnutzen der gefundenen Sicherheitslücken durch moderne Schutzmechanismen verhindert werden könnte,
        da diese Schutzmechanismen aus den oben beschrieben Gründen nur als letzte Verteidigungslinie dienen sollten.

        Auch eine Prüfung auf syntaktische Korrektheit der zu untersuchenden Quelltexte wird nicht vorgenommen.

        Stattdessen wird davon ausgegangen,
        dass es sich bei den zu prüfenden Quelltexten um syntaktisch korrekten Programmcode handelt,
        um den Arbeitsaufwand zu reduzieren und
        das Programm auf das Wesentliche
        -- die Erkennung von Sicherheitslücken
        -- zu beschränken.

        Weiterhin ist zu beachten,
        dass in einer statischen Analyse,
        welche beliebige Quelltexte untersuchen kann,
        eine Analyse der Programmlogik nicht möglich ist.
        Schwachstellen in der Programmlogik können aus diesem Grund nicht durch das Framework erkannt werden und
        es werden auch keine Versuche unternommen,
        solche zu erkennen.
        Eine derartige Analyse sollte stattdessen aus Sicht des Autors durch Unittests durchgeführt werden.

        So wird zum Beispiel nicht untersucht,
        ob ein Benutzer mit der Rolle
        \enquote{Benutzer} in der Lage ist,
        sich selbst die Rolle
        \enquote{Administrator} zu geben,
        wodurch seine unprivilegierte Rolle umgangen werden kann.

        Weiterhin ist es dem Autor nicht möglich,
        für sämtliche Programmiersprachen und
        Angriffsmuster Regeln zu schreiben.

        Aus diesem Grund wird versucht,
        anhand von Modulen für einzelne Programmiersprachen und
        einfach zu konstruierenden Regelsätzen eine Erweiterbarkeit zu gewährleisten.

        Eine Erstellung vollständiger Grammatiken und
        Regelsätze dagegen ist nicht Ziel der Ausarbeitung.
        Stattdessen werden vereinfachte Versionen erstellt,
        die zeigen sollen,
        wie eine Umsetzung im späteren Betrieb möglich sein wird.

        Da das Ziel der Ausarbeitung nicht die Erstellung eines vollständigen Parsers welcher auch die syntaktische Korrektheit der Eingabedateien prüfen kann,
        sondern die statische Analyse von Schwachstellen ist,
        scheint dieser Ansatz sinnvoll.

        Um die in der Motivation angesprochene manuelle Analyse zu vereinfachen,
        wird zusätzlich versucht,
        anhand verschiedener Metriken die Komplexität eines Codeabschnitts zu ermitteln.

        Da komplexerer Code naturgemäß auch ein höheres Fehlerpotenzial birgt,
        können hiermit entsprechende Abschnitte hervorgehoben werden,
        sodass die manuelle Analyse zielgerichteter und
        schneller durchgeführt werden kann.

\chapter{Test des Frameworks}
    Um die Praxistauglichkeit des Frameworks zu demonstrieren,
    wird im Folgenden gezeigt,
    wie man mittels Docker von beliebigen Systemen aus das Framework benutzen kann,
    selbst wenn das eigene System nicht die notwendigen Voraussetzungen
    -- installiertes Python 3.7 und
    libmagic
    -- erfüllt.

    Anschließend wird anhand eines Beispielreports gezeigt,
    welche Arten von Schwachstellen mit dem Framework gefunden werden können und
    deren Aufkommen im Quelltext stichprobenartig nachvollzogen.

    \section{Test auf verschiedenen Systemen mittels Docker}
        Bei Docker handelt es sich um eine freie Software zur Containervirtualisierung für Linux,
        Windows und
        MacOS.\cite{Docker2019}

        Der große Vorteil von Docker besteht darin,
        dass es hiermit möglich ist,
        Anwendungen,
        Konfigurationen und
        Abhängigkeiten gebündelt in einem Container auszuliefern,
        wodurch keine direkte Anpassung des Hostsystems mehr notwendig ist,
        um Software zu installieren.

        Stattdessen wird durch das Dockersystem eine vom Hostsystem isolierte Umgebung mittels
        \gls{CGroups},
        Namespaces und
        \gls{UnionFS} geschaffen.
        In Windows"=Systemen,
        wo diese Technologien nicht in dieser Form existieren,
        wird mittels Hyper"=V oder
        VirtualBox eine virtuelle Linux"=Maschine erstellt,
        in welcher die Umgebung wie beschrieben eingerichtet wird.\cite{Docker2019a}

        Docker baut seine Container mittels einer Beschreibung im sogenannten
        \foreignquote{english}{Dockerfile} auf.

        Für dieses Framework wird das folgende Dockerfile verwendet:

        \begin{lstlisting}[caption={Dockerfile für das entwickelte Framework}, language=Dockerfile, gobble=12]
            FROM python:3.7-alpine
            MAINTAINER Jonas A. Wendorf <jonas_wendorf@mail.de>
            RUN apk add libmagic
            COPY Projekt/ /app
            WORKDIR /app
            RUN pip install -r requirements.txt
            RUN pipenv install --ignore-pipfile
            ENTRYPOINT ["pipenv", "run", "python", "main.py"]
        \end{lstlisting}

        Die erste Zeile gibt hierbei an,
        dass das neue Image basierend auf einem Linux Alpine Image mit vorinstalliertem Python 3.7 gebaut werden soll.

        Linux Alpine hat hierbei den Vorteil,
        dass es sehr klein ist und
        somit auch die Dateigröße des resultierenden Images geringer hält,
        als dies mit größeren Basisimages wie Windows Server Core oder
        Debian Stretch,
        den anderen beiden offiziell im Docker Hub verfügbaren Basisimages für Python 3.7,
        möglich wäre.

        Auf der anderen Seite ist Linux Alpine so minimalistisch,
        dass in der dritten Zeile libmagic als Abhängigkeit nachinstalliert werden muss,
        da dieses standardmäßig nicht installiert ist.

        Die darauffolgende Zeile gibt an,
        dass das Projekt im Verzeichnis
        \filename{Projekt} liegt und
        nach
        \filename{/app} im Image kopiert werden soll.

        Dieses Verzeichnis wird dann auch als Arbeitsverzeichnis angegeben und
        die Installation der Python"=Pakete wird eingeleitet.

        Der in der letzten Zeile angegebene Entrypoint gibt an,
        dass beim Start des Containers direkt das Framework gestartet werden sollte,
        damit der Benutzer nicht erst im Container nach dem passenden Pfad suchen muss.

        Zum Bauen des Images kann anschließend auf einem Betriebssystem mit installiertem Docker der Befehl
        \lstinline{docker build . -t taintalyzing} eingegeben werden,
        um mit dem Dockerfile im aktuellen Verzeichnis ein Image zu erstellen und
        dieses mit dem Tag
        \enquote{taintalyzing} zu versehen.

        Anschließend kann das Dockerimage zum Beispiel mittels
        \lstinline{docker run --rm -v "$(pwd)/appdata":/appdata taintalyzing}
        gestartet werden,
        um den Container nach der Beendigung automatisch aufzuräumen und
        während der Laufzeit das Unterverzeichnis
        \filename{appdata} im aktuellen Verzeichnis in das Verzeichnis
        \filename{/appdata} im Container weiterzureichen.

        Von hier aus kann das Framework dann
        wie bei einer nativen Installation auf dem eigenen Betriebssystem benutzt werden mit dem einzigen Unterschied,
        dass nur Dateien im Verzeichnis
        \filename{/appdata} auf dem Zielsystem untersucht werden können.

        Wenn der Benutzer einen Report im
        \gls{HTML}"=Format erstellt,
        sollte außerdem darauf geachtet werden,
        dass die Dateien
        \filename{customize.js},
        \filename{logo.png} und
        \filename{style.css} ebenfalls aus dem Projektordner in den Ordner mit dem gespeicherten Report kopiert werden müssen.

        Für das ältere Docker Toolbox unter Windows ist außerdem wichtig zu beachten,
        dass standardmäßig nur Unterverzeichnisse von
        \filename{C:\textbackslash{}Users} in Container weitergegeben werden können.
        Weitere Pfade müssen manuell in VirtualBox für die virtuelle Maschine freigegeben werden.\cite{Procida2019}

    \section{Beispielreport und Auswertung}
        Für den folgenden Beispielreport wurde der Terminaldateimanager cfiles von Manan Singh mit Stand vom 17.~Januar 2019 genutzt
        (Commit
        \hash{a4e8ba74f2662f2df5a45deb3a4377936d5141fb}).\cite{Singh2019}

        Das Projekt wurde ausgewählt,
        da es zum einen in C geschrieben ist und
        einige der in dieser Sprache typischsten Sicherheitslücken aufweist,
        zum anderen,
        weil die besagten Sicherheitslücken mittlerweile korrigiert wurden und
        somit im Rahmen dieser Masterarbeit keine Zero"=Day"=Schwachstellen veröffentlicht werden,
        welche Benutzer des Projekts in Gefahr bringen könnten.

        Für die Auswertung wurde die folgende Regel zum Suchen nach Quellen genutzt:

        \begin{lstlisting}[caption={Regel für die Suche nach Quellen}, gobble=12]
            ---
            null:
                Methods:
                - Methodname: getenv
                Parameters: [null]
                Comment: Reads environment variables
        \end{lstlisting}

        Weiterhin wurden folgende Regeln eingesetzt,
        um Senken zu finden:

        \begin{lstlisting}[caption={Regeln für die Suche nach Senken}, gobble=12]
            ---
            null:
                Methods:
                - Methodname: sprintf
                Parameters: [null, null, $TAINT]
                Comment: No check for destination buffer size, use snprintf instead.
                - Methodname: strcpy
                Parameters: [null, $TAINT]
                Comment: No check for destination buffer size, use strncpy instead.
                - Methodname: system
                Parameters: [$TAINT]
                Comment: Executes arbitrary system commands.
        \end{lstlisting}

        Insgesamt wurden bereits mit nur diesen Regeln 24 Verschmutzungen und
        31 Senken gefunden.
        Weitere Schwachstellen existieren im Code,
        dieses Beispiel soll allerdings zeigen,
        wie typisch es für Entwickler ist,
        einen Fehler durchgängig zu machen,
        und
        wenn bei der Korrektur nur einer dieser Fehler übersehen wird,
        kann es zu fatalen Konsequenzen führen.

        Im Folgenden sind daher stichprobenartig einige der Befunde zusammen mit einem Vergleich des Quelltextes aufgeführt.

        Der vollständige Report im Reintext"= und
        \gls{HTML}"=Format befindet sich in den Anlagen.

        \foreignquote{english}{Analysis results for method ``init'' (lines 134 to 171).

        The following taints were detected:\\
        In line 145 a call with potentially user controlled input is made to sprintf.\\
        The following comment is linked to this sink: No check for destination buffer size, use snprintf instead.\\
        No sanitizer detected.\\
        Severity level: 100\%.}

        Im Code wird in Zeile 145 folgender Aufruf benutzt:

        \lstinline{sprintf(editor, "%s", getenv("EDITOR"));}

        Wie der Kommentar aus dem Report bereits andeutet,
        findet bei
        \lstinline{sprintf} keine Prüfung der Größe des Zielpuffers statt,
        entsprechend könnte ein Angreifer die Variable
        \lstinline{editor},
        welche in Zeile 53 als
        \lstinline{char editor[20]} deklariert wird,
        überfluten,
        indem er einen Eintrag mit mehr als 20 Zeichen verwendet.

        Die Senke wird dadurch zur Verschmutzung,
        dass im gleichen Aufruf zu
        \lstinline{sprintf} ein weiterer Aufruf zu
        \lstinline{getenv("EDITOR")} enthalten ist,
        wodurch die Umgebungsvariable
        \lstinline{EDITOR} eingelesen wird.

        Ein Angreifer,
        der Zugriff auf die Umgebungsvariablen hat,
        kann diese Variable entsprechend abändern und
        anschließend die Kontrolle über das System übernehmen.

        Ein ähnliches Problem befindet sich in Zeile 491.

        Hier meldet der Report

        \foreignquote{english}{Analysis results for method ``getImgPreview'' (lines 473 to 521).

        The following taints were detected:\\
        In line 491 a call with potentially user controlled input is made to sprintf.\\
        The following comment is linked to this sink: No check for destination buffer size, use snprintf instead.\\
        No sanitizer detected.\\
        Severity level: 100\%.}

        Im Code wird an dieser Stelle die folgende Zeile genutzt:

        \lstinline{sprintf(getdimensions_command,"echo -e \'5;%s'|/usr/lib/w3m/w3mimgdisplay",filepath);}

        Hier wird in den in Zeile 484 deklarierten Puffer
        \lstinline{char getdimensions_command[250]} die Eingabe von
        \lstinline{filepath} geschrieben,
        ohne die Größe von
        \lstinline{getdimensions_command} vorher zu prüfen.

        Die Senke wird in diesem Fall zur Verschmutzung,
        weil die Variablenverfolgung ergibt,
        dass
        \lstinline{filepath} in Zeile 473 als Parameter an die Methode übergeben wird und
        somit potenziell benutzerkontrolliert ist.

        Ein interessanter Fall ergibt sich in Zeile 519,
        in der Methode
        \lstinline{getImgPreview},
        wo
        \lstinline{system} genutzt wird,
        um einen Systembefehl auszuführen:

        \lstinline{system(imgdisplay_command);}

        Diese Stelle ist dadurch interessant,
        dass in der vorherigen Zeile die in diesem Aufruf verwendete Variable
        \lstinline{imgdisplay_command} durch
        \lstinline{sprintf} basierend auf dem vom Benutzer kontrollierten
        \lstinline{filepath} befüllt wird:

        \lstinline{sprintf(imgdisplay_command, "echo -e '0;1;%d;%d;%d;%d;;;;;%s\n4;\n3;' | /usr/lib/w3m/w3mimgdisplay", maxx+maxx/5, 8, width, height, filepath);}

        Der Aufruf zu
        \lstinline{system} wird allerdings nicht als Verschmutzung erkannt,
        da
        \lstinline{sprintf} in den Regeln nicht als Quelle markiert wurde.

        Dies demonstriert die zuvor beschriebene Notwendigkeit,
        auch Senken aufzunehmen,
        für die keine Quelle gefunden werden konnte,
        da es sein kann,
        dass lediglich eine entsprechende Regel für die Quellen fehlt.

        Die entsprechende Ausgabe im Report ist daher:
        \foreignquote{english}{In line 519 a call without any detected user controlled input is made to system.\\
        The following comment is linked to this sink: Executes arbitrary system commands.\\
        Severity level: 50\%.}

        Auch die Verfolgung von verwundbaren Methodenaufrufen lässt sich an diesem Projekt gut nachvollziehen.

        So wird in der Methode
        \lstinline{main} in Zeile 1061 die Methode
        \lstinline{getPreview} aufgerufen,
        welche wiederum in Zeile 616
        \lstinline{getArchivePreview} aufruft.

        Da in
        \lstinline{getArchivePreview} eine Verschmutzung erkannt wurde,
        allerdings keine benutzerkontrollierte Eingabe für den Aufruf von
        \lstinline{getPreview} in der
        \lstinline{main}"=Methode erkannt wurde,
        wird dieser Aufruf im Report lediglich als Senke und
        nicht als Verschmutzung ausgegeben:
        \foreignquote{english}{In line 1061 a call without any detected user controlled input is made to getPreview.\\
        The following comment is linked to this sink: Calls getArchivePreview from None.\\
        Severity level: 50\%.}

        Die Ausgaben,
        dass eine Methode
        \foreignquote{english}{from None} aufgerufen wird,
        bedeutet dabei jeweils,
        dass keine Objektzugehörigkeit erkannt wurde.

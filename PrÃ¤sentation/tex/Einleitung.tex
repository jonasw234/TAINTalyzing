\section{Einleitung}
    \begin{frame}{\secname}
        \begin{itemize}
            \item Sicherheitsprüfungen können statisch oder dynamisch sein
            \item Dynamisch: Eingaben werden am laufenden Programm vorgenommen
            \item Statisch: Mögliche Eingaben werden anhand des Quelltextes nachvollzogen
            \item Entwicklung eines Frameworks zur statischen Analyse von Quelltexten
        \end{itemize}
    \end{frame}

    \subsection{Motivation}
        \begin{frame}{\secname: \subsecname}
            \begin{itemize}
                \item Dynamische Sicherheitsprüfungen sind kostenintensiv und zeitlich limitiert
                \item Statische Sicherheitsprüfungen können kostengünstig in die Entwicklungspipeline aufgenommen werden
                \item Existierende statische Codescanner sind
                    \begin{itemize}
                        \item teuer
                        \item schwer zu erweitern
                        \item teilweise inkompatibel mit neuen Versionen einer Programmiersprache
                        \item nicht für alle Programmiersprachen verfügbar
                    \end{itemize}
            \end{itemize}
        \end{frame}

\section{Regelsätze}
    \subsection{Format}
        \begin{frame}[fragile]{\secname: \subsecname}
            \begin{itemize}
                \item Aufgebaut in YAML
                \item Unterschiedlich für Quellen und Senken
                \item[\textrightarrow] Senken haben optionalen Absicherungsblock
            \end{itemize}
            \begin{lstlisting}[gobble=16, escapechar=!]
                ---
                OBJEKTNAME:
                    Methods:
                    - Methodname: METHODENNAME
                      Parameters: [null, $TAINT, null]
                      Comment: KOMMENTAR
                      !\textcolor{gray!95}{Sanitizers:}!
                      !\textcolor{gray!95}{- OBJEKTNAME}!
                          !\textcolor{gray!95}{Methods:}!
                          !\textcolor{gray!95}{- Methodname: METHODNAME}!
                            !\textcolor{gray!95}{Parameters: []}!
                            !\textcolor{gray!95}{Comment: KOMMENTAR}!
            \end{lstlisting}
        \end{frame}

    \subsection{Beispielregel scanf}
        \begin{frame}[fragile]{\secname: \subsecname}
            \begin{itemize}
                \item \texttt{scanf} in C liest formatierten Text ein
                \item Erster Parameter ist Formatstring, darauffolgende Parameter sind Zielvariablen
            \end{itemize}
            \begin{lstlisting}[gobble=16]
                ---
                null:
                    Methods:
                    - Methodname: scanf
                      Parameters: [null, $TAINT]
                      Comment: Reads formatted input from stdin
            \end{lstlisting}
        \end{frame}

    \subsection{Beispielregel printf}
        \begin{frame}[fragile]{\secname: \subsecname}
            \begin{itemize}
                \item \texttt{printf} in C gibt formatierten Text aus
                \item Bei nur einem Parameter wird dieser als Formatstring erkannt
                \item Benutzer kann eigene Formatstrings einfügen, um Daten zu offenbaren und Speicher zu ändern
            \end{itemize}
            \begin{lstlisting}[gobble=16]
                ---
                null:
                    Methods:
                    - Methodname: printf
                      Parameters: [$TAINT]
                      Comment: Format string vulnerability.
            \end{lstlisting}
        \end{frame}

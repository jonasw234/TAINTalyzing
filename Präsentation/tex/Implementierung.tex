\section{Implementierung}
    \subsection{Vorgehensweise des Frameworks}
        \begin{frame}{\secname: \subsecname}
            \begin{enumerate}
                \item Quelltexte einlesen
                \item Passendes Modul auswählen
                \item Quelltexte parsen
                \item Regelsätze parsen
                \item Nach Schwachstellen suchen
                \item Codekomplexität analysieren
                \item Report erstellen
            \end{enumerate}
        \end{frame}

    % \subsection{Parsen der Quelltexte}
    %     \begin{frame}{\secname: \subsecname}
    %         \begin{itemize}
    %             \item PyParsing wird bevorzugt
    %             \item Grammatiken sind vergleichsweise einfach zu erstellen
    %             \item Eignet sich gut für einfachere Grammatiken
    %             \item[\textrightarrow] Für die Schwachstellensuche unwichtige Details werden ignoriert
    %             \item Andere Bibliotheken sind ebenfalls möglich, wenn die Ergebnisse an PyParsing angeglichen werden
    %         \end{itemize}
    %     \end{frame}

    % \subsection{Parsen der Regelsätze}
    %     \begin{frame}[fragile]{\secname: \subsecname}
    %         \begin{itemize}
    %             \item Regeln liegen im YAML-Format vor
    %             \item Werden mittels PyYAML geladen
    %             \item[\textrightarrow] Daten werden direkt in passenden Strukturen dargestellt
    %         \end{itemize}
    %         \begin{lstlisting}[gobble=16]
    %             {'OBJECT':
    %                 {'Methods':
    %                     [{'Methodname': 'METHODNAME',
    %                     'Parameters': [None, '$TAINT'],
    %                     'Comment': 'COMMENT'}]
    %                 }
    %             }
    %         \end{lstlisting}
    %     \end{frame}

    \subsection{Vorgehensweise bei der Suche nach Schwachstellen}
        \begin{frame}{\secname: \subsecname}
            \begin{enumerate}
                \item Parser ermittelt Methoden innerhalb der Datei
                \item Findet alle Variablen, die innerhalb der Methode verwendet werden
                \item Sucht nach möglichen Quellen, Senken oder Absicherungen
                \item Vergleicht die Eingabeparameter mit der Variablenliste
                \item[\textrightarrow] Schließt hardcodierte Methodenaufrufe aus
                \item Verfolgt Variablen zurück
                \item Informiert das Framework über neu hinzugekommene verwundbare Methoden
                \item Wiederholung bis keine neuen Ergebnisse geliefert werden
            \end{enumerate}
        \end{frame}

    \subsection{Variablenrückverfolgung}
        \begin{frame}[fragile]{\secname: \subsecname}
            \begin{itemize}
                \item Verfolgung der Variable »test« im Aufruf von \texttt{printf}
            \end{itemize}
            \begin{lstlisting}[gobble=16, escapechar=!]
                #include <stdio.h>
                !\hl<6>{int main(int argc, char * argv[]) \{}!
                    !\hl<5>{char * foo = argv[1];}!
                    !\hl<4>{char * bar = foo;}!
                    !\hl<3>{char * test = bar;}!
                    !\hl<2>{printf(test);}!
                }
            \end{lstlisting}
        \end{frame}

    \subsection{Exklusive Kontrollstrukturen}
        \begin{frame}[fragile]{\secname: \subsecname}
            \begin{itemize}
                \item Kontrollstrukturen, die wechselseitig exklusive Pfade erzeugen
                \item Beispiel: if/else
            \end{itemize}
            \begin{lstlisting}[gobble=16, escapechar=@]
                #include <stdio.h>
                int main(int argc, char* argv[]) {
                    @\colorbox{yellow}{if (0) \{}@
                        @\colorbox{yellow}{test();}@
                    } @\colorbox{green}{else if (2) \{}@
                        @\colorbox{green}{printf(argv[1]);}@
                    } @\colorbox{cyan}{else if (3) \{}@
                        @\colorbox{cyan}{printf("hallo\textbackslash{}n");}@
                    } @\colorbox{brown}{else \{}@
                        @\colorbox{brown}{printf("bye!\textbackslash{}n");}@
                    }
                    return 0;
                }
            \end{lstlisting}
        \end{frame}

%     \subsection{Komplexitätsprüfung}
%         \begin{frame}{\secname: \subsecname}
%             \begin{itemize}
%                 \item Nutzt zyklomatische Komplexität
%                 \item \( v(G) = e - n + 2 \)
%                 \item \( n \) sind die Knoten, \( e \) die Kanten
%                 \item Jede Anweisung ist ein Knoten und eine Kante
%                 \item Exklusive Kontrollstrukturen sind ein Knoten und zwei Kanten
%                 \item Schleifen sind ein Knoten und drei Kanten (Einstieg, Wiederholung, Ende)
%                 \item Gibt an, wie viele Testfälle maximal benötigt werden, um die Methode komplett zu testen
%                 \item Eher geringe Aussagekraft, aber gut als Indikator für mögliche Probleme geeignet
%             \end{itemize}
%         \end{frame}
